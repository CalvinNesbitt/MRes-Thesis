\section{MAOOAM}

\subsection{Model Outline}

The Modular Arbitrary Order Ocean Atmosphere Model (MAOOAM) is a spectral, coupled atmosphere ocean model that allows for an arbitrary number of modes to be kept in the spectral decomposition \cite{Cruz2016}. The atmosphere component is two-layer quasi-geostrophic and is coupled to a QG shallow-water ocean layer. We note that that the coupling is both thermal and mechanical. The mechanics for the model are governed by forced QG equations:

\begin{align} \label{Mechanical equations}
 \pdv{t} \left( \laplacian \psi_a^1 \right) + J \left(\psi_a ^1 , \laplacian \psi_a^1 \right) + \beta \pdv{\psi_a^1}{x} = -k'_d \laplacian \left(\psi_a^1 - \psi_a^3 \right) + \frac{f_0}{\Delta p} \omega
\end{align}

\begin{multline}
 \pdv{t} \left( \laplacian \psi_a^3 \right) + J \left(\psi_a ^3 , \laplacian \psi_a^3 \right) + \beta \pdv{\psi_a^3}{x} \\ = k'_d \laplacian \left(\psi_a^1 - \psi_a^3 \right) - \frac{f_0}{\Delta p} \omega - k_d \laplacian \left(\psi_a^3 - \psi_o \right) 
\end{multline}

\begin{multline}
 \pdv{t} \left( \laplacian \psi_o - \frac{\psi_o}{L^2_R} \right) + J \left(\psi_o , \laplacian \psi_o \right) + \beta \pdv{\psi_o}{x} \\ = -r \laplacian \psi_o + \frac{C}{\rho h} \laplacian \left(\psi_a^3 - \psi_o \right).
\end{multline}

\noindent Whilst the thermodynamics are governed by an energy balance scheme:

\begin{align}\label{Thermodynamic equations 1}
\gamma_a \left( \pdv{T_a}{t} + J(\psi_a, T_a) - \sigma \omega \frac{p}{R}\right) &= - \lambda(T_a - T_o) + \epsilon_a \sigma_B T_a^4 - 2 \epsilon_a \sigma_B T_a^4+ R_a \\ 
\gamma_o \left( \pdv{T_o}{t} + J(\psi_o, T_o) \right) &= - \lambda(T_o - T_a) - \sigma_BT_o^4 + \epsilon_a \sigma_B T_a^4 + R_o \label{Thermodynamic equations 2}
\end{align}

\noindent An explanation of each term in equations \ref{Mechanical equations} - \ref{Thermodynamic equations 2} can be found in table \ref{MAOOAM Table}. The quartic long term radiation fluxes in \ref{Thermodynamic equations 1} and \ref{Thermodynamic equations 2} are linearised around spatially uniform temperatures $T_o^0$ and $T_a^0$. Then through use of the hydrostatic relation and ideal gas law the authors reduced the independent variables to $\psi_a = \frac{\psi_a^1 + \psi_a^3}{2}$, $\psi_o$, $\delta T_a$ and $\delta T_o$. Here $\psi_a$ is the atmospheric barotropic stream function whilst $\delta T_a$ and $\delta T_o$ are temperature anomalies resulting from the aforementioned linearisation.\\

\noindent Each independent variable is then approximated by a truncated Fourier expansion where different basis functions, determined by the boundary conditions, are used for the atmosphere and ocean. We note that the atmosphere is assumed to be zonally periodic with no-flux in the meridional direction whilst the ocean is a box with no-flux along all boundaries.  The model then solves for the coefficients of the Fourier expansions by solving a series of $2(n_a + n_o)$ ODEs. Here $n_a$ and $n_o$ are the number of basis functions used for the Fourier expansion in the atmosphere and ocean respectively. We note that both $n_a$ and $n_o$ can be freely changed by the model user.

\begin{table}
\begin{center}
\scalebox{0.75}{\begin{tabular}{ |c c c c| } 
 	\hline
 	Term & Meaning & Term & Meaning \\
 	\hline
 	$\psi_a^1$ & $250$ hPa atmospheric streamfunction & $T_a$ & Atmospheric temperature  \\ 
 	$ \psi_a^3$ & $750$ hPa atmospheric streamfunction & $T_o$ & Ocean temperature  \\ 
 	$\psi_o$ & Ocean streamfunction & $\omega = \dv{p}{t}$ & Vertical velocity  \\ 
 	$n$ & Aspect ratio & $L_R$ & Reduced Rossby deformation radius  \\ 
 	$L$ & Characteristic length scale & $\rho$ & Ocean density  \\ 
 	$f_0$ & Coriolis parameter & $\sigma _B$ & Boltzmann constant  \\ 
 	$\lambda$ & Atmosphere-ocean heat transfer & $\sigma$ & Atmosphere static stability  \\ 
 	$r$ & Bottom ocean friction & $\beta$ & Rossby parameter  \\ 
 	$d = \frac{C}{\rho h}$ & Drag coefficient & $R$ & Ideal gas constant  \\ 
 	$C_o$ & Ocean short wave insolation & $\gamma_o$ & Ocean heat capacity  \\ 
 	$C_a$ & Atmosphere short wave insolation & $\gamma_a$ & Atmosphere heat capacity  \\ 
 	$k_d$ & Atmospheric layer friction & $T_a^0$ & Atmosphere base temperature  \\ 
 	$k'_d$ & Atmosphere ocean layer friction & $T_o ^0$ & Ocean base temperature  \\ 
 	$h$ & Ocean depth & $\epsilon_a $ & Atmospheric emissivity  \\ 
 	\hline
	\end{tabular}}
\caption{Terms appearing in MAOOAM model equations.} \label{MAOOAM Table}
\label{table:1}
\end{center}
\end{table}

\subsection{Model Dynamics \& The Effects of Resolution}

A fixed low spectral resolution predecessor to MAOOAM was found to exhibit a low frequency variation in the form of a coupling between atmospheric and oceanic modes \cite{Vannitsem2015} \cite{Vannitsem2015a}. This is an interesting feature of the model due to the open question on the origin of mid-latitude low frequency atmospheric variability that has been suggested by time series analysis of observed data \cite{Trenberth1990}.\\

\noindent In \cite{Cruz2016} the robustness of the aforementioned LFV is examined for increased spectral resolution, with the authors finding that the signal strength initially decreases as the resolution is increased before returning. Of course this dependence of the dynamics on the resolution naturally leads to questions of what the is appropriate truncation level for convergence to be achieved. Examining the variance as resolution is increased, the authors find that the atmospheric components of the model converge relatively quickly whilst the ocean is much slower. The explanation given for this by the authors is the need to resolve the Rhines scale $L_{Rh}$ which seperates wave dominated QG regimes from turbulent ones \cite{Vallis2017}. Of course the problem this presents is that the resolution required to resolve $L_{Rh}$ is such that the model becomes computationally heavy and can no longer be considered light weight. This point is further built upon in \cite{DeCruz2018} where the MAOOAM Lyapunov exponent spectra are examined. Here the atmospheric instability is shown to be correctly represented whilst the central manifold dynamics, which the authors associate to the long time-scale ocean dynamics, are found not to be correctly represented at low or intermediate resolutions. These results highlight the need to investigate the role of resolution in MAOOAM, possibly through the use of the Lyapunov formalism. For example one would like to know if there is an optimal low resolution that captures the essential dynamics or to better understand the structural reasons behind the changing dynamics as resolutions increases.

\subsection{Experimental Setting for Constructing $\chi$}

We now outline the experimental set up we will consider in MAOOAM to test the utility of linear response theory. We note that the following is identical in design to previous work such as \cite{Lucarini2017a}, \cite{Gritsun2017} and \cite{Lucarini2011}. We have chosen to look at the effect of perturbations $\delta$ to short wave radiative forcing $C_a$ on the zonally averaged wind speed $\expval{\psi}$.\\ 

\noindent Our first step will be to apply perturbations of varying strength $\delta$, $2\delta$, $3\delta$ etc. to asses the linearity of the models response. Given that we do observe linear response, we will then construct the response function $\chi$. This will be done by applying a perturbation taking the form of a Heaviside step function $\Theta$ and estimating the linear response in $\expval{\psi}$ via:

\begin{align}
\expval{\psi}^1_\delta \approx \frac{\expval{\psi}_{C_a + \delta}(t) - \expval{\psi}_{C_a - \delta}(t)}{2}
\end{align}

\noindent As in \cite{Lucarini2017a} we will then be able to invert Ruelle's formula for $\chi$ when we have $\Theta$ time modulation. One would then aim to test the use of the constructed $\chi$ by comparing further model experiments with alternatively time modulated perturbations to the results predicted by $\chi$.