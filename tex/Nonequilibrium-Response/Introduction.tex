The results of the previous section apply for systems that are near to equilibrium. Of course the climate is not such a system \cite{Peixoto1984}. However, Ruelle was able to extend the previous results for a wider class of systems that are far from equilibrium \cite{Ruelle}. To understand precisely the class of systems that Ruelle was able to prove these extended results for, we will introduce some basic notions from the ergodic theory of chaos \cite{Eckmann1985}, which is a fairly modern view of dynamical systems theory. Specifically, we'll first introduce characteristic exponents that allow us to quantify the notion of chaotic dynamics. Then we will define the notion of a hyperbolic invariant set, which is a well understood class of dynamically invariant objects. Finally, we will outline the chaotic hypothesis that serves as a working assumption in practical scientific work involving chaotic dynamics.  