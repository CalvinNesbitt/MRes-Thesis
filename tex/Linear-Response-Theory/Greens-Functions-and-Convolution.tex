% Convolution integral summary

\section{Response Functions} \label{section: Response Functions}

We are interested in the response of a state variable $X$ to an external weak forcing $h$, where $h$ is coupled to $X$ i.e. they are conjugate. Throughout we will assume $h$ is small enough that the response $\expval{X}_h$ is linear in h. We will define then define the response function $\chi(t, t')$ as:

\begin{align}
\expval{X}_h = \chi * h
\end{align}

We begin us motivating response functions. Suppose we have some linear system: 

\begin{align}
\mathcal{L}x(t) = F(t). \label{linear-eqn}
\end{align}

Here $\mathcal{L}$ is a differential operator that governs the dynamics of the system and $F$ represents an external forcing, whilst the solution $x(t)$ represents the response of our system to the forcing. For example for the classical damped harmonic oscillator we have:

\begin{align*}
\mathcal{L} x = \dv[2]{x}{t} + \gamma \dv{x}{t} + \omega ^2 x.
\end{align*} 

\noindent Finding the solution $\mathcal{L}^{-1}$ for an arbitrary forcing $F$ is in general not achievable. Fortunately, the principle of superposition means we need only find the response $\chi$ to an impulse forcing $\delta$ in order to find the response in general. Indeed recall the following fact:

\begin{thm}
If $\mathcal{L}\chi(t) = \delta(t)$, then the solution to equation \ref{linear-eqn} is given by $x(t) = \chi * F = \int_0^t \chi(t, \tau) F(\tau) \dd \tau$.
\end{thm}


\subsection{Frequency Dependent Response}

Looking at the Fourier transform of $\chi(t)$ we obtain %add appendix of livi physical interpretation of each part + that response is local in frequency
