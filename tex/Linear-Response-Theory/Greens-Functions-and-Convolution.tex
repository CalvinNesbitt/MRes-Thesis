% Greens functions and Convolution integral summary

\section{Response Theory Background Reading: Convolution and Green's functions}

We give an overview of the theory of green's functions and the convolution integral for LTI systems.

\subsection{Response via Convolution}

From a signals point of view we can think of the linear, time invariant ODE:

\begin{align*}
\mathcal{L}x(t) = s(t) 
\end{align*}

as having two parts, the input signal $s(t)$ and the response $x(t)$. From this point of view the operator $\mathcal{L}^{-1}$ will give us the response of the system to given input.

\begin{align*}
\text{input} \to \mathcal{L}^{-1} \to \text{response}.
\end{align*}

In particular if we can figure out the response to a $\delta$ impulse (this is the green's function that plays the role of $\mathcal{L}^{-1}$), we can then use a convolution sum to figure out the general response. This idea works due to the linearity of the system. Explicitly we can write the response as:

\begin{align*}
x(t) = \int_{0}^{t} s(\tau) \chi (t - \tau) \dd \tau
\end{align*} 


where $\chi$ is the green's function.




