% Linear Response Theory Chapter Introduction

The classical theory of linear response allows us to determine how the macroscopic properties of a thermodynamic system respond to a weak external field that forces the system \cite{Dorfman1999}. We present results that have been found to hold for near-equilibrium systems, that is we will assume our thermodynamic system reaches an equilibrium and even in the presence of the field it remains "near" to this equilibrium \cite{LiviPoliti2017} \cite{Tong}. Moreover, for ease of exposition, we will be examining classical systems described by a Hamiltonian formalism. We begin by introducing response functions before discusing their basic mathematical properties such as the Kramers-Kronig relations that are a consequence of causality. A key aspect of near-equilibrium linear response is the similarity \textit{on average} between a thermodynamic systems decay to equilibrium and it's fluctuations around that equilibrium \cite{Reichl1999}. This will be discussed through the Fluctuation dissipation theorem that we will derive here. Finally, before beginning our journey we note that our hope is that this section will serve as a parallel to seciton \ref{section: Ruelle} where linear response is explored for far from equilibrium systems such as the climate. 