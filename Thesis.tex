\documentclass[11pt,letterpaper,twoside,openright]{book}

\title{Response Theory in Multiscale GFD Systems - MRes Thesis}
\date{\today}
\author{Calvin Nesbitt}

\usepackage{style} % Read Style Package

\begin{document}
%
%\maketitle
%
%\tableofcontents 
%
%\cleardoublepage

%\chapter{Introduction}

% Linear Response Theory
\chapter{Linear Response Theory}
% Linear Response Theory Chapter Introduction

The classical theory of linear response allows us to determine how the macroscopic properties of a thermodynamic system respond to a weak external field that forces the system \cite{Dorfman1999}. We present results that have been found to hold for near-equilibrium systems, that is we will assume our thermodynamic system reaches an equilibrium and even in the presence of the field it remains "near" to this equilibrium \cite{LiviPoliti2017} \cite{Tong}. Moreover, for ease of exposition, we will be examining classical systems described by a Hamiltonian formalism. We begin by introducing response functions before discusing their basic mathematical properties such as the Kramers-Kronig relations that are a consequence of causality. A key aspect of near-equilibrium linear response is the similarity \textit{on average} between a thermodynamic systems decay to equilibrium and it's fluctuations around that equilibrium \cite{Reichl1999}. This will be discussed through the Fluctuation dissipation theorem that we will derive here. Finally, before beginning our journey we note that our hope is that this section will serve as a parallel to seciton \ref{section: Ruelle} where linear response is explored for far from equilibrium systems such as the climate. 
% Convolution integral summary

\section{Response Functions} \label{section: Response Functions}

The experimental set up is as follows, we consider a thermodynamic system consisting of $N$ particles obeying the laws of classical mechanics. We assume the system has reached an equilibrium and can thus be described by the canonical ensemble \cite{Greiner1995}. We will be interested in measuring the response of a state variable $X$ to a weak perturbation $h$, where $h$ and $X$ are conjugate variables. Throughout we will assume $h$ is small enough that the response $\expval{X}_h$ is linear in $h$. The response function $\chi(t, t')$ will give us the response to a forcing at time $t'$. We define $\chi(t, t')$ through:

\begin{align} 
\expval{X}_h &= \chi * h\\ \label{ChiDef}
&= \int_{- \infty}^{\infty} \dd t' \chi(t - t') h(t')
\end{align}

\noindent Where does this definition come from? Well, suppose we have some linear system: 

\begin{align}
\mathcal{L}x(t) = F(t). \label{linear-eqn}
\end{align}

\noindent Here $\mathcal{L}$ is a differential operator that governs the dynamics of the system and $F$ represents an external forcing, whilst the solution $x(t)$ represents the response of our system to the forcing. For example for the classical damped harmonic oscillator we have:

\begin{align*}
\mathcal{L} x = \dv[2]{x}{t} + \gamma \dv{x}{t} + \omega ^2 x.
\end{align*} 

\noindent Finding the solution $\mathcal{L}^{-1}$ for an arbitrary forcing $F$ is in general not achievable. Fortunately, the principle of superposition means we need only find the response $\chi$ to an impulse forcing $\delta$ in order to find the response in general. Indeed recall the following fact:

\begin{thm}
If $\mathcal{L}\chi(t) = \delta(t)$, then the solution to equation \ref{linear-eqn} is given by $x(t) = \chi * F = \int_0^t \chi(t, \tau) F(\tau) \dd \tau$.
\end{thm}

\noindent Thus if we are able to find such a $\chi$ we are able to determine the response for any forcing. The take away, finding $\chi$ gives us the linear response $\expval{X}_h$.


\subsection{Frequency Dependent Response}

Looking at the Fourier transform of the response function, dentoed $\chi(\omega)$ we can obtain some physical insight. Indeed each of the real and imaginary parts of the complex valued $\chi(\omega)$ has a physical interpretation \cite{LiviPoliti2017}. First of all through a simple manipulation one can show:

\begin{align}
\Im{\chi(\omega)} = - \Im{\chi(-\omega)}. 
\end{align} 

\noindent Namely, the imaginary part is odd and is therefore aware of the direction of time. This is the part of the response that contains information about dissipative processes \cite{LiviPoliti2017}. In contrast $\Re{\chi(\omega)}$ is even and insensitive to the arrow of time, we will refer to this as the \textit{reactive} part of the response function. Finally when we note that the definition of $\chi(t)$ is the form of a convolution, we can use the convolution theorem to write a relation between the Fourier transforms of \ref{ChiDef}:

\begin{align}
\expval{X(\omega)}_h &= \chi(\omega) h(\omega).
\end{align}

\noindent Thus, in the context of linear response theory, forcing at a given frequency excite response at the same frequency.

\section{Properties of Response Functions}

Having introduced response functions, we will now describe their basic properties. First up we will examine physical constraints we will wish to impose on them and the mathematical consequences of these. In particular we will want them to be time invariant and causal. Building on the implications of causality we will then derive the Kramers-Kronig relations which allow us to relate the imaginary and real parts of the Fourier transformed response function $\chi(\omega)$. We will briefly mention the physical interpretation of each part before we move on. The following are standard results that can be found along with further details in \cite{LiviPoliti2017} and \cite{Reichl1999}.

\subsection{The Implications of Causality}

Cause must precede effect. This condition of causality is a natural requirement of our response functions, namely our system can't respond until it is first forced:

\begin{align}
\chi(t - t') = 
\begin{cases*}
\chi(t - t') & if $t -t' > 0$.\\
0 & otherwise.
\end{cases*}
\end{align}

This constraint limits the locations of poles $\chi(\omega)$.

\subsection{The Kramers-Kronig Relations}

Since $\chi$ is analytic in the upper half of the complex plane one can relate its imaginary and real parts.
\section{The Fluctuation Dissipation Theorem}

%% Chaos and Ergodicity
%\chapter{Chaotic Dynamical Systems}
%\section{Characterising Chaos}

\cite{Eckmann1985a} \cite{Dorfman1999}\cite{Ott2002} % LYAPUNOV EXPONENTS, STABLE/UNSTABLE Manifolds
%\section{Hyperbolic Dynamics}

\cite{Gallavotti1995a} \cite{Dorfman1999} % Splitting tangents space, toral automporphisms, arnold cat map
%\section{The Chaotic Hypothesis}

\cite{Gallavotti1995a} \cite{Dorfman1999} % Could possibly go in above section
%
%
%% Ruelle's Response Theory
%\chapter{Response Theory for Nonequilibrium Systems}
%\section{SRB Measures}

\cite{Eckmann1985a} %Need more references
%%\input{the role of geometry}
%
%% Response Theory in GFD 
%\chapter{Applications in Geophysical Fluid Dynamics}
%\section{The Fluctuation Dissipation Theorem in Climate}

\cite{Gritsun2017}

% Experimental Results

% Bibliography
\clearpage 
\bibliographystyle{alpha}
\refstepcounter{section} 
\addcontentsline{toc}{chapter}{References}
\bibliography{/Users/cfn18/Documents/BibTex-Files/MRes.bib} %Bibtex file location

\end{document}

