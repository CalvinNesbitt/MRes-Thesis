\documentclass[11pt,letterpaper,twoside,openright]{book}

\title{Response Theory in Multiscale GFD Systems - MRes Thesis}
\date{\today}
\author{Calvin Nesbitt}

\usepackage{style} % Read Style Package

\begin{document}
%
%\maketitle
%
%\tableofcontents 
%
%\cleardoublepage

%\chapter{Introduction}

% Linear Response Theory
\chapter{Linear Response Theory}
% Linear Response Theory Chapter Introduction

The classical theory of linear response allows us to determine how the macroscopic properties of a thermodynamic system respond to a weak external field that forces the system \cite{Dorfman1999}. We present results that have been found to hold for near-equilibrium systems, that is we will assume our thermodynamic system reaches an equilibrium and even in the presence of the field it remains "near" to this equilibrium \cite{LiviPoliti2017} \cite{Tong}. Moreover, for ease of exposition, we will be examining classical systems described by a Hamiltonian formalism. We begin by introducing response functions before discusing their basic mathematical properties such as the Kramers-Kronig relations that are a consequence of causality. A key aspect of near-equilibrium linear response is the similarity \textit{on average} between a thermodynamic systems decay to equilibrium and it's fluctuations around that equilibrium \cite{Reichl1999}. This will be discussed through the Fluctuation dissipation theorem that we will derive here. Finally, before beginning our journey we note that our hope is that this section will serve as a parallel to seciton \ref{section: Ruelle} where linear response is explored for far from equilibrium systems such as the climate. 
% Convolution integral summary

\section{Response Functions}

We begin with the linear system: 

\begin{align}
\mathcal{L}x(t) = F(t). \label{linear-eqn}
\end{align}

Here $\mathcal{L}$ is a differential operator that governs the dynamics of the system and $F$ represents an external forcing, whilst the solution $x(t)$ represents the response of our system to the forcing. For example for the classical damped harmonic oscillator we have:

\begin{align*}
\mathcal{L} x = \dv[2]{x}{t} + \gamma \dv{x}{t} + \omega ^2 x.
\end{align*} 

\noindent Finding the solution $\mathcal{L}^{-1}$ for an arbitrary forcing $F$ is in general not achievable. Fortunately, the principle of superposition means we need only find the response $\chi$ to an impulse forcing $\delta$ in order to find the response in general. Indeed recall the following fact:

\begin{thm}
If $\mathcal{L}\chi(t) = \delta(t)$, then the solution to equation \ref{linear-eqn} is given by $x(t) = \chi * F = \int_0^t \chi(t, \tau) F(\tau) \dd \tau$.
\end{thm}


Classical linear response theory deals with small perturbations to equilibrium systems. We further assume that $\chi(t, \tau)$ is invariant under time translations \cite{Tong} that is: 

\begin{align*}
\chi(t, \tau) =  
\begin{cases}
\chi(t - \tau) \text{ if } t > \tau \\
0 \text{ otherwise. } 
\end{cases}
\end{align*}

\section{Properties of Response Functions}
\subsection{Causality}
\subsection{Kramers-Kronig Relations}
\section{The Fluctuation Dissipation Theorem}

%% Chaos and Ergodicity
%\chapter{Chaotic Dynamical Systems}
%\section{Characterising Chaos}

\cite{Eckmann1985a} \cite{Dorfman1999}\cite{Ott2002} % LYAPUNOV EXPONENTS, STABLE/UNSTABLE Manifolds
%\section{Hyperbolic Dynamics}

\cite{Gallavotti1995a} \cite{Dorfman1999} % Splitting tangents space, toral automporphisms, arnold cat map
%\section{The Chaotic Hypothesis}

\cite{Gallavotti1995a} \cite{Dorfman1999} % Could possibly go in above section
%
%
%% Ruelle's Response Theory
%\chapter{Response Theory for Nonequilibrium Systems}
%\section{SRB Measures}

\cite{Eckmann1985a} %Need more references
%%\input{the role of geometry}
%
%% Response Theory in GFD 
%\chapter{Applications in Geophysical Fluid Dynamics}
%\section{The Fluctuation Dissipation Theorem in Climate}

\cite{Gritsun2017}

% Experimental Results

% Bibliography
\clearpage 
\bibliographystyle{alpha}
\refstepcounter{section} 
\addcontentsline{toc}{chapter}{References}
\bibliography{/Users/cfn18/Documents/BibTex-Files/MRes.bib} %Bibtex file location

\end{document}

